\subsection{Facilities, Equipment and Other Resources: UNL}

NOTE: THIS IS SUPERSEEDED BY THE FacilitiesEquipmentResourcesUNL.doc FILE.



\subsubsection{Facilities}
\paragraph{Laboratory:} We have a 500 square foot clean room for pixel detector
assembly and testing.  It is equipped with an computer-controlled
Aerotech gantry that can be used for precision pick-and-place
assembly; custom tooling for the gantry has been designed in
collaboration with Purdue as well as the LabView code for control; an
automatic and a manual wirebonder; a test stand that can be used to
test the CMS pixel readout chip and detector modules; a manual
microprobe station for ASIC testing; and assorted equipment such as
oscilloscopes, microscopes and power supplies.  A high-bay area is
also available for the construction of larger components of
particle-physics experiments.

\paragraph{Computing:} The US CMS Tier-2 computing cluster is part of UNL's
Holland Computing Center (HCC).  The cluster currently has more than
23~kHS05 of processing resources and can host about 1.2~PB of CMS data.
HCC also operates the Tusker cluster in Omaha, which has several thousand
processing cores; these resources are available to the HEP group on an
opportunistic basis.  All group members have access to a desktop or laptop
machine for personal computing.

\paragraph{Office Space:} All faculty members have individual offices in Jorgensen
Hall, the newly built home of the UNL Department of Physics and Astronomy, and on-site
graduate students have a shared office.  The group also has offices for
students, postdocs and visiting faculty at Jorgensen, the LPC and at
CERN.

\paragraph{Other:} Research is conducted at CERN in Geneva, Switzerland (site of
the CMS Experiment); Fermi National Accelerator Laboratory in Batavia, IL
(site of the D0 Experiment); the Pierre Auger Observatory in Mendoza
Province, Argentina; \fixme{and also list neutrino facilities}.  The UNL
HEP group also owns one PolyCom videoconferencing unit and has access to
another one that is owned by the Department of Physics and Astronomy.

\subsection{Major Equipment}
The project makes use of the CMS detector at CERN and has access to the equipment and laboratory infrastructure at PSI.

\subsection{Other Resources}
Secretarial support and business office support are provided by the
University of Nebraska-Lincoln Department of Physics and Astronomy. The
department hosts excellent instrument and electronics shops, with full time
permanent staff members, which is available for projects of the high energy
physics group at reduced in-house hourly labor rates.

