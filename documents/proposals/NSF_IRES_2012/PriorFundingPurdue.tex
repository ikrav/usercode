
\subsection{Purdue University}

Daniela Bortoletto has recently received NSF support for hosting the
Conference Undergraduate Women In Physics (CUWiP) at Purdue in
2011. CUWiP took place simultaneously at Purdue University, The
University of California Santa Cruz, North Carolina State University,
Duke University, and MIT. The Purdue conference reached 113
undergraduate women from the Midwest, bringing them together to
establish a network of young women in physics. These women had the
opportunity to learn about the frontiers of physics research and
possible careers in physics with the goal of encouraging them to
pursue higher degrees in physics. They interacted extensively with
successful women physicists and had the chance to discuss challenges
facing women in physics. For many of the students who presented talks
and posters, the conference was their first opportunity to attend a
scientific meeting and present their work. For others, the lab tours
gave them their first view of active research labs at the forefront of
the physical sciences. The president of Purdue University France
Cordova, was the keynote speaker. Her presentation was broadcasted
from Purdue to all sites.

In 2010 NSF supported Daniela Bortoletto to host a workshop on “New
Opportunities in Energy Research” covering solar energy, hydrogen
fuels and fuel cells, biofuels, nuclear, catalysis and bio-catalysis,
battery and capacitor technologies, sustainable materials, etc. The
workshop focused on Systems Integration, Design of New Materials, and
Computational Efforts in Energy. The workshop was extremely successful
and resulted in the submission of a white paper.

Bortoletto was a recipient of the NSF Career and Career Advancement
Award. The NSF support was fundamental to establish the Purdue
Particle Physics Microstructure Detector Facility (P3MD), a suite of
three modern, fully equipped, laboratories that has been described as
“a nationally significant facility” and a “national treasure”. P3MD
has been critical to the training of undergraduate and graduate
students in the development of state of the art silicon sensors.

The Purdue Group has also received critical funding from the NSF for
the Maintenance and operation of the Pixel system and for R\&D for the
development of the phase 1 pixel system. The outcome of the NSF
support has been the smooth operation of the pixel system despite the
challenging LHC environment. Purdue is playing a leadership role in
developing the mechanical design for the disk of the upgraded
detector. Our engineer, Arndt co-leads the joint engineering effort at
Purdue and Fermilab to complete the conceptual design solution for the
Phase 1 upgrade. He completed initial engineering to balance CO2 flow
conditions and adjust the insertion scheme for BPIX/FPIX upgrade
detector systems. Purdue has proposed a baseline design and process
flow for pick-and-place assembly of upgrade FPIX modules. Arndt, aided
by temporary employees, has integrated a Gantry positioning system
with camera/optics, pattern recognition, custom vacuum pick-up and
glue dispensing tools for semi-automated pick-and-place assembly of
upgrade FPIX modules.  Purdue collaborates with other groups in the
design and evaluation of ultra-radiation silicon sensors for the Pixel
Upgrade.
