
\subsection{University of Kansas}

In addition to the NSF PIRE and MRI activities under the direction of
principal investigator Alice Bean, the KU high energy physics group is
most recently supported through the NSF EPP grant 0970004
Experimental Particle Physics Research with the focus on work with the
CMS collaboration.  With CMS, the KU group has been involved in
several analysis projects: including the inclusive $b$ cross
section\cite{bcross}, single top production\cite{singletop}, search
for signals in the the two photon channel for Gauge-Mediated
Supersymmetry-Breaking and Universal Extra Dimensions\cite{dmsearch},
and the search for the Higgs boson in the decay $H\rightarrow
b\overline{b}$\cite{higgsbb}.  The group is also concentrates on the
CMS pixel detector research which includes monitoring and calibrating
the detector as well as helping to design and build the Phase I pixel
upgrade.  Over the past 4 years, 15 KU undergraduate students have
worked with the CMS project.  Bean also leads the multimedia outreach
program {\it Quarked!  Adventures in the Subatomic Universe} which
continues to expand its offerings on the popular website
www.quarked.org.  Visits to the Quarked! website continue to grow,
with more than 72,000 unique visitors in 2011. In addition to online
resources, the project includes two hands-on museum education programs
for schools at the KU Natural History Museum (KUNHM) that have reached
more than 5,000 participants to date.  Assessment results from this
can be found here\cite{quarked}.  Bean with KU physics professor Judy
Wu (PI) also obtained NSF support through the Communicating Research
to Public Audiences program (award number 1065789) for {\it Adventures
  at Nanoscale: Superconductivity}. This project supports the creation
of a 6 minute animated video about conductors and superconductors
using the Quarked! characters.



