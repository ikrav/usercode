
\section{Logistics}

% General support from the Swiss side
%
The institutes involved in this proposal have had significant
experience with placing researchers in Switzerland, including
arranging undergraduate and graduate student stays at ETHZ and
PSI. The primary location for IRES activities will be ETHZ. The ETHZ
side will provide the needed research equipment and facilities for
silicon pixel detector research and development.  They will also
provide: the necessary space, video conference facilities, housing
location assistance, and administrative assistance.  Here, the
administrative assistance
% But we are not yet sure about invoice and billing
includes helping to obtain travel and work visas, invoice payments and
billing back to U.S. institutions, as well as living arrangement
assistance.

% Loding arrangements
%
During the PIRE grant, the logistics for having students live and
perform research in Switzerland were well understood.  A furnished
apartment was rented near PSI and help was given to administer the
financial aspects of this arrangement by our Swiss collaborators.
Expectations are to either to continue to rent this apartment (which
is an hour away from ETHZ by public transport) or obtain one closer to
the ETHZ campus.  Internet connectivity in the apartment will be
assured for students to participate in meetings or to communicate with
friends and family using freely based software such as Skype.  The
public transport system in Switzerland is superb and individual car
transport is not needed.  The cost of living in Switzerland is more
than in the U.S. so extra cost of living expenses are requested in the
budget.

% Health insurance
%
Some of the students will be resident in Switzerland for longer than
three months and are required to carry acceptable health insurance.
While students may be able to use their current U.S. health insurance
coverage to satisfy this policy, past experience has indicated that
many U.S. health insurance plans are not accepted by the Swiss
authorities.  Therefore we have budgeted for Swiss health insurance
plan coverage. We have also budgeted for emergency assistance and
evacuation/repatriation insurance.  These costs and procedures are
well understood by the institutes involved.

% Language and cultural adjustment
%
English is the language of choice for CMS collaboration meetings, and
all Swiss citizens learn English as part of their schooling.
Therefore, while the primary language in the Zurich area is German,
the student participants are not expected to have any problems with a
language barrier at both the professional and daily life settings. In
the course of the routine research work on the CMS experiment, all
faculty from the participating U.S. institutions who are this proposal's
PIs, co-PIs, and the future mentors of the IRES students have either
frequently visited or lived for extended times in
Switzerland. Therefore, the U.S. mentors will be able to provide a
wealth of advice that will ease the cultural adjustment for IRES
students.

