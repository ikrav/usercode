
\subsection{University of Puerto Rico at Mayaguez}

The High Energy Physics Group at the University of Puerto Rico's
Mayaguez Campus (UPRM) has been funded continuously by the Department
of Energy since 1994 to participate in experiments at Fermilab and
more recently at the Cornell LEPP and CERN.

The group has been working on CMS for six years and we has made
significant contributions to the experiment in the construction of the
forward pixel subdetector and the development of simulation, hardware
monitoring and offline tracking software.  More recently it has also
undertaken hardware and simulation projects related to the design of
an upgraded pixel system for the planned increase to the LHC
luminosity.  Those projects are in addition to a program of data
simulation and analysis carried out at the group's CMS Tier3 computing
cluster at Mayaguez.

However, it is important to point out that UPR's contributions to the
field of high energy physics go beyond the scientific. We refer here
to the quality of the nurturing we provide to a group of students who
would be lost to HEP were it not for the existence of the UPR HEP
group.  Of twenty MS graduates in the HEP group in the last eighteen
years, four have obtained their PhDs, seven are currently in PhD
programs at major institutions (four in HEP), one is engaged in
industrial research in the US, one runs a Linux cluster in a research
setting and seven are teaching at the college level.  The interactions
with HEP labs (such as Fermilab) have also had an important impact on
the undergraduate population.  Since these relationships began, forty
undergraduates have participated in summer internships at US HEP labs.
In addition, through involvement in QuarkNet, we are bringing HEP to
K-12 classrooms all across Puerto Rico reaching a population which
typically has very little direct exposure to vanguard science.

UPR is a member of the PIRE consortium which has been working with PSI
and ETHZ during the last five years.  Four graduate students and four
undergraduate students have spent extended periods at PSI (one year in
the case of the graduate students and two periods of two summer months
for most of the undergraduates).  Two students have written their MS
thesis based mainly on their PSI research and are now in PhD programs.
The other two will receive their MS title this semester.  All have
presented their work at major conferences.

Dr. Ramirez will be the UPR PI. He has been a UPR HEP faculty member
since 2003 mostly devoted in CMS to developing advanced vertexing
algorithms and to establishing the UPR Tier3 cluster.  He will be
assisted by Dr. Angel Lopez who will soon retire from his present
position of UPR professor and HEP group leader.  Dr. Lopez will act in
the role of Adjunct Professor and provide his experience with the PIRE
project.


