\documentclass[12pt]{article}
%\usepackage{epsfig}
\usepackage{graphicx}
\usepackage{graphics}
\usepackage{wrapfig}
\usepackage{subfigure}
\usepackage{setspace}

\oddsidemargin   0.00 in
\evensidemargin  0.00 in
\textwidth       6.50 in
%\voffset         0.00 in
\topmargin       0.0 in
\textheight      9.00 in

\begin{document}

\pagestyle{empty}

\begin{center}
{\large\bf Project Summary}
\end{center}

This IRES proposal requests a three-year funding for supporting
research of several groups of students from UNL, Purdue University,
KSU, and KU on elementary particle physics with the leading experts in
the field from several institutions in the Zurich (Switzerland)
area. Each of the three years, a cohort of three to five students
total will work at the collaborating Swiss institutions for three to
eight months. The international partners who will play the key role in
the activities of this proposal include: Prof. Rainer Wallny and
Dr. Andrey Starodumov from the Eidgenoessische Technische Hochschule
(ETH), Prof. Roland Horisberger from Paul Scherrer Institute and ETH,
and Prof. Benjamin Kilminster from the University of Zurich. The
primary location where the students will be stationed is ETH.

The research plan for the students is formulated in the context of the
international CMS experiment at the Large Hadron Collider (LHC), and
concerns the pixel detector of charged particles at CMS. Silicon pixel
detectors belong to the category of instruments which lead to new
frontiers in measurement techniques and hence in physics. At the LHC,
close to the interaction point, no other detector instrument is
capable to cope with the high density and rate of particle tracks as
pixel detector can because of their fine granularity. The CMS
collaboration plans to replace the present pixel detector which was
conceived over 10 years ago
% and designed for a maxim luminosity of
%$1\times 10^{34}$ cm$^{-2}$s$^{-1}$ and 25 ns bunch crossing 
with a new
light-weight pixel detector equipped with a more powerful digital
readout.

This IRES proposal will allow many students to conduct research
projects with state of the art pixel detectors including detector
calibrations with test beams, radiation hardness tests of the sensors
and the readout chip, and development of silicon module testing and
certification procedures. The interaction with ETH, PSI and Zurich
will be fundamental since the new readout chip for the Phase~1 CMS
pixel system has been developed by our Swiss collaborators.

The research plan will be well integrated in the overall CMS Phase~1
upgrade work. The Swiss senior personnel Prof. Wallny,
Prof. Horisberger, and Prof. Kilminster has committed to identify
projects for each student appropriate in scale and complexity. At ETH,
the students will be mentored and guided by Prof. Wallny and
Dr. Starodumov, and supervised by a U.S. post-doc stationed there.


{\bf Intellectual merit:} The global-engagement of U.S. science and
engineering students capable of performing in an international
research environment at the forefront of science and engineering is
critical to the advancement of the complex experiments required to
answer the most advanced problems. The work of IRES students will
contribute to achieving the goals of the CMS experiment: understanding
the nature of the mechanism of the Electroweak symmetry breaking and
finding phenomena outside of the standard model of particle
physics.This work will also result in the advancement of the silicon
detector technology used for high precision measurements in many
fields of science and engineering.

{\bf Broader impact:}
The expertise the students will acquire will be of great use in
setting up and running analogous kinds of projects at their
Universities and even in other scientific fields. The participants
will benefit themselves both from exposure to scientific research in
international environment as well as from months-long immersion into
a European culture. This proposal, continuing the tradition of its
predecessor, PIRE, will continue to focus on diversity and recruiting
Hispanic and female student participants. Additionally, the consortium
has Hispanic and female principal investigators in this proposal.

\end{document}