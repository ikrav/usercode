
\section{Collaboration between US Universities and
Swiss institutions}

Building an apparatus of the scale of the CMS detector is a
challenging undertaking and requires significant investment into
designing, building, and commissioning every subsystem. The present
CMS Pixel Detector described in the introductory sections is designed
and built by more than a dozen Universities and national labs. The
labor is divided according to the available local resources.
Traditionally, Swiss institutions build the Barrel Pixel Detector
while the US institutions are responsible for the Forward Pixel
Detector. The final integration takes place at CERN. Among the
institutions both on the Swiss and the US side, the labor is further
split so that module production, module testing, layer/disk assembly,
and so forth, take place at different locations according to the
expertise and resources of each contributing party.

Over the years of building, commissioning, and operating the presently
running Pixel Detector in CMS, this model of collaboration has lead to
close ties between the participating institutions. All of the US
institutions who are part of this collaborative proposal have a long
history of involvement in the CMS Forward Pixel Detector project. From
the European side, the three leading institutions in the Barrel Pixel
Detector counterpart who are also our international partners in this
proposal are the the EHTZ, PSI, and University of Zurich.

One of the most important aspects of the collaboration between the US
Universities of Midwest, Puerto Rico, and the Swiss institutions
concerns the research and education of undergraduate and graduate
students. The three-year IRES project, which would start in 2013,
continues to build on collaborations established previously with an
NSF Partnerships in International Research and Education (PIRE)
grant. For the PIRE grant (2007-2013), the research on the CMS Pixel
Detector was centered at the PSI near Zurich. Dozens of undergraduate
and graduate students from five US institutions (KU, KSU, UIC, UNL,
and UPRM) spent at least three months in Switzerland.  There was also
significant collaboration with the ETHZ group which included
Prof. Wallny, Prof. Christoph Grab, and Dr. Andrey Starodumov who
previously was employed by PSI but now manages the pixel research lab
at ETHZ.

For this IRES proposal, the work will be centered at ETHZ with day
trips expected by researchers to PSI. We are also building the
collaboration to take advantage of the planned coherence of the Swiss
CMS Pixel Upgrade consortium which includes ETHZ, PSI, and the
University of Zurich with newly hired Professor Benjamin Kilminster
(previously a collaborator on CDF with several of the PIs of this
proposal). The IRES proposal will use the momentum acquired by PIRE as
most of the senior participants from the Swiss sides, as well four out
of the original five PIRE institutions, are part of the proposal. The
Purdue University, who is not part of the original PIRE, has its own
strong ties with the Swiss parties, and have been sending exchange
undergraduate and graduate students to ETHZ and PSI over the last
years for exactly the same research projects.

% XXX Possibly this educational piece needs to be moved to
% a more appropriate place. If there is one.
ETHZ is well known as a leader in technical education.  With the above
mentioned PIRE grant, eleven students (both undergraduate and
graduate) enrolled for course credit during the semester.  For this
program, it is anticipated that graduate students who stay during the
Fall or Spring semesters may decide to take courses at ETHZ while they
are performing their research.  With the PIRE grant there were direct
enrollments by students as well as direct exchange students through
both KU and UIC.  The students were able to find challenging and
useful courses presented in English that were successfully completed.
In particular, the {\it Statistical Methods and Analysis Techniques in
  Experimental Physics} course presented by CMS collaborator Christoph
Grab was taken by two U.S. graduate students who needed extra
coursework once they arrived in Switzerland to complete their progress
towards their Ph.D.  We anticipate the IRES students will be able to
take advantage of the now existing administrative framework if they
need to take coursework at ETHZ.
