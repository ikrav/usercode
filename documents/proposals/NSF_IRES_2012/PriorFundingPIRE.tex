
\section{Results from prior funding}

\subsection{Results from the NSF PIRE grant and related MRI}
\label{sec:PIRE}

%%XXX-TO-BE-PROPERLY-DESCRIBED-XXX

%%NSF OISE-0730173, PIRE: Collaborative research with the Paul Scherrer
%%Institute and Eidgenoessische Technische Hochschule on Advanced Pixel
%%Silicon Detectors for the CMS detector 10/1/2007-9/30/2013

%%0960128 NSF MRI-R2: Development of a Pixel Detector for the Upgraded
%%CMS Experiment (04/15/2010-9/30/2013).

Three out of four institutions collaborating on this proposal (UNL,
KSU, and KU) currently enjoy the support from NSF OISE-0730173, PIRE:
Collaborative research with the Paul Scherrer Institute (PSI) and
Eidgenoessische Technische Hochschule on Advanced Pixel Silicon
Detectors for the CMS detector (10/1/2007-9/30/2013). The PIRE group,
comprising UNL, KU, KSU, UIC, and UPRM, has teamed together with our
Swiss colleagues for an international research and educational
experience. A total of 40 undergraduate and graduate students had
significant research experiences mostly located near PSI.

The CMS collaboration has now published over 160 refereed
publications.  On July 4, 2012, the CMS collaboration announced the
observation of a Higgs like particle. The entire collaboration
contributed to this remarkable result and the students with the PIRE
collaboration were also a part of this celebration. The journal paper
announcing these results is expected in Aug, 2012.  The references
\cite{1-8} show particular CMS papers in which the PIRE group
contributed significantly.  A major focus of the work, however, was to
understand and improve particle detector instrumentation.  Therefore,
several independent publications on these studies also
produced\cite{9-15}.  More CMS and instrumentation publications are
expected.  Topics of research included: radiation damage to silicon
detectors, electronics chip design and optimization, electronics
radiation damage and single event upset characterization, calibration
and testing of pixel modules, studies of materials for use in detector
design, data acquisition board design, high rate throughput
electronics studies, tracking in particle detectors, and X-ray
detector systems optimization.

In addition to the research portion of the PIRE grant, significant
efforts went into helping to improve institutional study abroad
activities.  A total of 11 students took courses and passed them
successfully at ETHZ during the grant.  A new exchange program was
created between UIC and ETHZ.  Undergraduate students were found to
excel in both the research and educational activities.  There was also
an effort to attract minorities and underrepresented students.  There
were 10 females, 10 hispanics, and one American native among the
postdocs and students who participated in this grant.

The PIRE group was expanded to include Rice and Rutger�s Universities
and an NSF Major Research Instrumentation grant NSF MRI-R2:
Development of a Pixel Detector for the Upgraded CMS Experiment
(04/15/2010-9/30/2013) is currently being completed.  Here, the focus
is on building the CMS Phase I pixel upgrade project.  PSI is a major
partner with this grant providing matching funding.  A full chain
readout system is being constructed that will culminate with the pilot
pixel system to be installed into CMS during the 2013-2014 shutdown.
A few modules are being constructed using the new digital readout and
token buffer manager chips to take data before the full installation
of the phase I pixel detector.  The 400 Mbps digital readout has been
studied already with partial testing before the pilot detector
installation.
