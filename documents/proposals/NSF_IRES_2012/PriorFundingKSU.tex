
\subsection{Kansas State University}


The Kansas State University is one of the collaborating institutions for 
the PIRE and MRI grants
mentioned above and that provide funding for research and development on 
the CMS pixel detector upgrade. 
The KSU group has also recently received funding through NSF EPSCoR (NSF 
EPS-000561574 - 
``Towards New Discoveries at the Energy Frontier" (?) - Ivanov's First 
Award) that helped to support 
physics analyses activities at CDF and CMS: the $t\bar{t}$ cross section 
measurements and searches 
for up-type and down-type fourth generation quarks. 

The KSU high energy physics group receives most of its research support 
through the DOE HEP office.
Notable accomplishments over past three years include leadership in 
developing efficient methods 
for identifying electrons and photons (Maravin and post-doc Khalil), 
electroweak 
physics analyses (Bolton, Maravin, graduate students Chakaberia, 
Svintradze ), 
top physics and searches ( Ivanov, post-doc Khalil, graduate student 
Makouski).

%
%The KSU group has received funding through NSF EPSCoR (NSF EPS-9550487-  
"Kansas 
%Alliance for  Cosmology-Interdisciplinary Astrophysics")  and  NSF  MRI  
(NSF  PHY-0116649- 
%"Development of the Silicon Vertex Detector for the Higgs Search at the 
Tevatron Collider").
%The NSF EPSCoR funding  helped establish the K-State Electronics Design 
Laboratory  and to 
%start a synergistic program in Cosmology at KSU. The NSF MRI grant 
supported the successful
%construction of the Layer 0 upgrade for the D0 detector at Fermilab. 
%
%KSU receives most of its research support through the DOE HEO office, and 
it has received 
%(in collaboration with KU) funding from the DOE EPSCoR program.  Notable 
accomplishments
%over the past five years have included leadership of projects in D0 to 
design and build data path 
%electronics  for  the  D0  SMT  tracker  and  its  layer  0  upgrade 
(retired faculty  member Sidwell, 
%Stanton,  and  Taylor);  to  improve  operational  efficiency  of  the  
SMT  (post-doc  Harder);  to 
%develop efficient methods for identifying b-quarks (post-doc Rizatdinova) 
and photons (Maravin 
%and  post-doc  Bandurin);  and  to  complete  analyses  in  electroweak  
physics  (Bolton,  Maravin, 
%Harder,  and  graduate student  Ferapontov),  top  quark  physics 
(Rizatdinova), strong  interaction 
%physics  (Bandurin,  Bolton,  and  graduate  student  Ahsan),  and  
searches  for  supersymmetry 
%(Bolton and graduate student Shamim).
%On the CMS  experiment, KSU  has tested  all VHDI circuits for the 
forward  pixel  detector 
%(Bolton and Taylor); has written many of the programs describing the 
forward pixel geometry for
%offline software  and simulation  (post-doc  Onoprienko);  has  designed  
and  implemented  major 
%portions of the database software for the pixel detector (post-doc Wan);  
has  co-developed the 
%remote web-based monitoring tool WBM used widely  by  all  of CMS (Wan);  
has  developed 
%offline  reconstruction  software  for  photon  and electron 
identification  and  reconstruction 
%(Maravin and Bandurin); and has co-coordinated the U.S. efforts in 
several CMS commissioning
%projects (Maravin). 

KSU outreach activities include a 22-teacher QuarkNet center that serves 
small rural
Kansas schools, and an interdisciplinary program at KSU (Bolton) called 
the Center for Understanding of 
Origins that advances teaching and scholarship of  cosmology, evolutionary 
biology, and related 
sciences. 
