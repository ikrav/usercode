\section{Research projects for IRES participants}
\label{sec:projects}

As previously mentioned, the construction of the Phase~1 upgrade to
the CMS pixel detector will be shared among groups from Europe and the
United States.  The innermost layers of the barrel pixel detector will
be built by the Swiss consortium of PSI, ETHZ and the University of
Zurich, who are recognized for their expertise in pixel detector
design and construction.  The United States is responsible for the
forward pixel detector construction.  The design of this new upgrade
has been done in such a way as to minimize the differences in
components needed for the forward versus barrel detectors.  The hybrid
modules used for the forward and barrel detectors are essentially the
same, for example.  This proposal aims to continue building the bridge
between these Swiss and US groups through an exchange of expertise in
pixel detector design, testing, characterization and construction.

The US students placed at ETHZ will be working with our Swiss
colleagues on the following projects during stays of approximately
eight months after which they will return to their home institutions
bringing with them the skills and knowledge gained.
\begin{itemize}
\item Test beams
  \begin{itemize}
  \item Pre-production detector modules will be studied in test beams
    at PSI, Fermilab and other facilities
  \item Track reconstruction efficiencies and resolutions will be
    measured
  \item The performance of the detector will also be validated in high
    rate data taking similar to what will be present in the inner
    regions of CMS during high luminosity operations of the LHC
  \end{itemize}
\item Radiation hardness tests
  \begin{itemize}
  \item Both the readout chip and the sensor have been designed to be
    tolerant of harsh radiation environment of the LHC
  \item These rad-hard design principals need to be verified and
    characterized by comparing the pre and post-irradiation
    performance in test beams and on the bench
  \end{itemize}
\item Module test procedures
  \begin{itemize}
  \item Building on previous experience doing something similar to
    what was done for the current pixel detector, detailed procedures
    to characterize the performance of the pixel modules need to be performed
  \item The intrinsic noise of the readout chip plus sensor
    combination will be measured
  \item The procedure called ``trimming'' to normalize all pixel
    thresholds across the readout chip to the same value will be
    optimized to ensure the lowest thresholds possible
  \item High rate performance of the system will be validated with
    x-ray test stands.
  \item These same x-ray stands can also be used to produce
    mono-energetic x-rays to be used for energy calibrations of the
    readout chip
  \end{itemize}
\end{itemize}

The research plan for the students will be well integrated in the
overall CMS Phase 1 upgrade work, and the Swiss senior personnel has
committed to identify projects for each students appropriate in scale
and complexity. The mentoring faculty from the U.S. institutions and
the Swiss side will regularly meet and discuss the progress with the
projects and the performance of each student.

At ETHZ, the students will be mentored and guided by the Swiss faculty
and senior researchers, and supervised by a U.S. post-doc stationed
there. Upon completion of their visit, the students will return to
continue these activities at the home institutions.  The expertise the
students will acquire working with the leading Swiss pixel detector
experts, and familiarity with the test stands and electronic equipment
specifically developed for the Phase~1 Pixel Detector, will be of
great use in setting up and running analogous kinds of projects at
their Universities.
% Students supervised by a US postdoc at ETHZ will collaborate with our
% Swiss colleagues on the projects outlined above and will then return
% home to continue such activities at the home institutions.  
The goal is to build teams of new pixel detector experts and to
transition from construction to installation and commissioning of the
new detector after the funding cycle of this grant is finished.

The tangible outcome for the students themselves, in addition to
acquiring new skills and knowledge, as well as experience in conducting
research with international partners, will be the "hardware chapter"
in the Ph.D. thesis for doctoral students, or a M.S. thesis for
students pursuing the M.S. degree.
