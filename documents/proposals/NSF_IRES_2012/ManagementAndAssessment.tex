
\section{Recruitment, management, and mentoring}

% Recruitement
%
Faculty and researchers across the consortium will recruit from a
broad pool of qualified students interested in and already engaged
with research related to the area of focus. Each of the participating
U.S. Universities has a strong HEP group with a sizable number of
faculty, post-doctoral researchers, and graduate students already
working on research topics related to the CMS experiment. Providing
these graduate students with direct experience of a research work in
person with international partners is invaluable to their development
as scientists. Thus, graduate students working toward a Ph.D. or an
M.S. in particle physics will constitute the primary pool of the IRES
participants. Additionally, involvement of talented undergraduate
students is also expected. Each of the participating institutions has
at least several undergraduate researchers working on HEP projects at
any given time, these undergraduates will be the second pool of
candidates for participation in IRES.

It is expected that each year, a cohort of three to five students will
be sent to ETHZ by the collaborating Universities, typically with one
or two students from each institution. The stays of the UNL, KU, KSU,
and Purdue students in the Zurich area will be synchronized to have
maximum overlap, typically starting in May and lasting until the end
of summer in some cases and through the Fall semester in others. It is
expected that at least several US students are working together at any
given time, with the population peaking in the summer months. This
synchronized schedule will provide for the best research experience
for the participants. It will also make the management and logistics
easier to handle. For the three years of IRES, three different cohorts
of this type are planned.

% Diversity
%
The consortium is committed to increasing the diversity of physics
students.  
%%% This is a good sentence, but after reshuffling pieces it does
%%% not quite fit.
%A key strength of this program will be the partner
%institutions themselves. 
%%%% The sentence below is already used above
%Faculty and researchers across the consortium
%will recruit from a broad pool of qualified students interested in and
%already engaged with research related to the area of focus.  
With the previous similar grant, the PIRE, there was a particular
focus on recruiting Hispanic and female students and postdocs where a
total of 11 minority and 10 females participated. 
%Eighteen percent of
%the science and engineering Ph.Ds granted to Hispanics in the U.S. are
%to students who have passed through UPR as either an undergraduate or
%graduate student.  Four Hispanic graduate students from UPRM obtained
%their Master's degrees with PIRE and proceeded to PhD studies. 
This focus will continue in the IRES program. In addition to diversity of
the recruited student participants, the consortium has Hispanic and
female principal investigators in this proposal.

% Management
%
The UNL will be the lead institution for this grant and will
coordinate the student visits and research activities. The effort will
be centered at ETHZ, however smaller projects may take the
participants also to PSI and University of Zurich. All three Swiss
institutions commit to providing scientific mentorship as well as the
necessary equipment and facilities.  Each U.S. institution will assure
supervision and mentoring for all of the participating students from
their institution, contribute to the pixel research efforts both in
Switzerland and at their home institution, commit to increasing their
infrastructure for science students to research abroad, and integrate
the students from this grant into their overall research program on
CMS.  Faculty at both the Swiss and U.S. institutes will be charged with
mentoring all of the students, providing a stronger support network
across the collaboration.

The faculty and senior researchers at the Swiss side who will be
involved in mentoring of IRES participants have been identified
already.  At ETHZ, Professor Rainer Wallny leads the research group
which includes Senior Researcher Andrey Starodumov who manages the
pixel laboratory.  The PSI research group is headed by Roland
Horisberger who also is a professor at ETHZ and lead the Swiss effort
for the previous PIRE project.  At the University of Zurich, new
professor Benjamin Kilminster has just been hired and is excited by
the prospect of collaborating with students and faculty from the U.S.

Each student will additionally have an assigned faculty mentor at
their home institution with ultimate accountability. The U.S. faculty
mentors will be provided with a modest travel budget in order to visit
ETHZ each year for duration of several weeks.  The CMS collaboration
has developed the video conferencing and other tools needed to
communicate with researchers remotely.  There will be weekly
institute, collaboration, and consortium meetings scheduled using
personal computer based communications that will ensure regular
guidance of the IRES students by their U.S. mentors. We also plan
to have a U.S. postdoctoral researcher (supported by other funding)
located at ETHZ to assist the students in their research program.

